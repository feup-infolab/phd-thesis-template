%!TEX root = ../TeseDoutoramento.tex

\newglossaryentry{application profile}
{
  name=Application Profile,
  description={An Application Profile is defined by the \gls{DCMI} as a declaration of which metadata terms an organization, information resource, application, or user community uses in its metadata~\cite{Chan2006}}
}
\newglossaryentry{persistent identifier}
{
  name=Persistent Identifier,
  description={A persistent identifier ``is an association between a character string and an object. Objects can be files, Parts of files, persons, organizations, abstractions, etc. The identifiers can be \glspl{URN}, \glspl{DOI}, or \glspl{URI}''~\cite{Citation2011}}
}
\newglossaryentry{controlled vocabulary}
{
  name=Controlled Vocabulary,
  plural=Controlled Vocabularies,
  description={A controlled vocabulary is an established list of standardized terminology for use in indexing and retrieval of information. An example of a controlled vocabulary is subject headings used to describe library resources~\cite{oecd:controlledvocabulary}}
}
\newglossaryentry{CMS}
{
  name=Content Management System,
  description={A Content Management System is a computer application designed to allow the creation, editing, publishing, modification and maintenance of content from a central interface, as well as its related metadata~\cite{Mauthe2005}}
}
\newglossaryentry{representation information}
{
  name=Representation Information,
  description={Representation information is, according to the OAIS recommendations~\cite{Systems2002}, additional information that accompanies a physical or digital object, and that is necessary for expressing its meaning. For physical objects, it can record physically observable attributes of the object. For digital objects, it is necessary to adequately map the sequences of bytes or characters that make up the digital object into their higher-level meanings. As such, representation information is that information necessary to correctly interpret a sequence of bits as a usable digital object, such as a document or a file}
}
\newglossaryentry{ontology}
{
  name=Ontology,
  plural=Ontologies,
  description={Ontologies are, in short, ``specifications of conceptualizations''~\cite{gruber:ontologydefinition}. Ontologies are closely related to the semantic web~\cite{Horrocks2008} and are excellent tools for defining relevant concepts of a domain (the types of things and relationships between those things). They also provide higher-level representations of datasets because they do not rely on the structure or document syntax features such as nesting or numbering of elements~\cite{BULT:BULT283}. They not only provide richer semantic representations but can also be used to infer knowledge over existing sets of relationships between different entities, something that can play an important role in the retrieval of research information (``semantic search''). Ontologies evolve through their reuse, making it one of the goals to aim for when designing one~\cite{Bontas05casestudies}. The common understandings that arise from that reuse are the basis for interoperability~\cite{Annamalai03guidelinesfor}}
}
\newglossaryentry{BLAST}
{
  name=BLAST,
  plural=BLASTs,
  description={BLAST is an algorithm designed to calculate similarities for biological sequences, used in genetics and biology research. More information can be found in the original publication~\cite{Altschul1990}}
}
\newglossaryentry{Open Government}
{
  name=Open Government,
  description={Open government is the governing doctrine which holds that citizens have the right to access the documents and proceedings of the government to allow for effective public oversight~\cite{Lathrop:2010:OGC:1840977}}
}
\newglossaryentry{content negotiation}
{
  name=Content negotiation,
  description={HTTP Content negotiation allows, among other things, a client to request information from a server in a specific format, if available, using only features present in the HTTP protocol itself. According to the RFC 7231, section 3.4~\cite{Fieldinget.al.}: ``when responses convey payload information, whether indicating a success or an error, the origin server often has different ways of representing that information; for example, in different formats, languages, or encodings. Likewise, different users or user agents might have differing capabilities, characteristics, or preferences that could influence which representation, among those available, would be best to deliver.  For this reason, HTTP provides mechanisms for content negotiation.''}
}
\newglossaryentry{open data}
{
  name=Open Data,
  description={Open Data refers to Data available under open conditions. According to the \gls{OKF}, the concept of Open, when applied to a knowledge work, means that ``The work shall be available as a whole and at no more than a reasonable reproduction cost, preferably downloading via the Internet without charge. The work must also be available in a convenient and modifiable form.''~\cite{10.1371/journal.pbio.1001195}}
}
\newglossaryentry{ACID}
{
  name=ACID,
  description={Atomicity, Consistency, Isolation, Durability is a set of properties that guarantee that database transactions are processed reliably~\cite{Haerder:1983:PTD:289.291}}
}
\newglossaryentry{triple store}
{
  name=Triple Store,
  plural=triple stores,
  description={A triple store is designed to store and retrieve identities that are constructed from triplex collections of strings (sequences of letters). These triplex collections represent a subject-predicate-object relationship that more or less corresponds to the definition put forth by the RDF standard\footnote{\url{http://www.w3.org/2001/sw/Europe/events/20031113-storage/positions/rusher.html}}
  }
}
\newglossaryentry{foreign key}
{
  name=Foreign Key,
  plural=foreign keys,
  description={In relational databases, a foreign key is a field, or a collection of fields in one table that uniquely identifies a row in another table~\cite{Coronel2009}
  }
}

\newglossaryentry{data curator}
{
  name=Data curator,
  plural=data curators,
  description={A data curator is a person responsible for performing data curation, which is ``the activity of, managing and promoting the use of data from its point of creation, to ensure it is fit for contemporary purpose, and available for discovery and re-use. For dynamic datasets this may mean continuous enrichment or updating to keep it fit for purpose. Higher levels of curation will also involve maintaining links with annotation and with other published materials''\cite{Lord2003}
  }
}
\newglossaryentry{serendipity}
{
  name=serendipity,
  description={In recommender systems literature, ``serendipity is a measure of how surprising the successful recommendations are. For example, if the user has rated positively many movies where a certain star actor appears, recommending the new movie of that actor may be novel, because the user may not know of it, but is hardly surprising. Of course, random recommendations may be very surprising, and we therefore need to balance serendipity with accuracy.''\cite{Shani2011}
  }
}
\newglossaryentry{Nepomuk File Ontology}
{
  name=Nepomuk File Ontology,
  description={An ontology designed for ``describing native resources available on the desktop''. See \url{http://www.semanticdesktop.org/ontologies/2007/01/19/nie/}
  }
}